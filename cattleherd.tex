\documentclass{article}
\usepackage[top=1in,bottom=1in,left=1in,right=1in]{geometry}
\usepackage{listings}
\usepackage{hyperref}
\usepackage{color}
\usepackage[pdftex]{graphicx}
\usepackage{xspace}


\title{Models of FMDV in a Herd of Cattle}
\author{Drew Dolgert}
\date{\today}

\begin{document}
\maketitle
\abstract{There is significant data on the progress of FMDV within individual
cattle. This data describes both the progress of the disease and its variability
among individuals according to various effects. Obtaining multiple samples of
similar data for herds of infected cattle is impractical or inhumane, so
models extend understanding at an individual level to a herd level. This article
examines a family of models for FMDV progress within a herd in order to
understand how both the average progress of disease and an estimate of its
variability.}

\section{Introduction}
Focus on a putative herd of beef cattle of the same breed and similar life stage.
Build several models of varying complexity.


\section{FMDV progress within an individual}
\begin{figure}
\centerline{\begin{tabular}{lll}
\hline
FMD stage & time interval [days] & Poisson & Distribution \\ \hline
Latent &  $t_1-t_0$ 3.59 & Weibull ($\alpha=1.782$, $\beta=3.974$) \\
Subclinical & $t_2-t_1$ & 2.04 & Gamma ($\alpha=1.222$, $\beta=1.672$) \\
Incubation & $t_2-t_0$  & 5.9 & Log logistic ($\gamma=0$, $\beta=5.3$, $\alpha=4.02$) \\
Infectious &  $t_3-t_1$ & 4.39 & Gamma ($\alpha=3.969$, $\beta=1.107$)
\end{tabular}}
\caption{From ``FMDV serotype O infection in domestic animals,''\label{fig:compartments}}
\end{figure}

Mardones et al.\ define five observations for the progress of FMDV within
an individual, exposure at $t_0$, first reported positive sample
at $t_1$, first clinical sign at $t_2$, last reported positive sample at $t_3$,
and last clinical sign at $t_4$. These observations are the boundaries
of compartments for compartmental models, as described in Fig.~\ref{fig:compartments}.


This work will represent the state of an individual with two
states, the disease state and the clinical sign.
The disease states are latent, infectious, and recovered.
The clinical signs are not clinical or clinical. Together, these states
represent all of the compartments. An individual can become clinical
only once infectious. As an approximation given the paper's data, the individual
will become subclinical again once recovered.

\section{Herd Models}



\end{document}
